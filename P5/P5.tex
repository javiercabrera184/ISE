\input{preambulo.tex}

%----------------------------------------------------------------------------------------
%	TÍTULO Y DATOS DEL ALUMNO
%----------------------------------------------------------------------------------------

\title{	
\normalfont \normalsize 
\textsc{{\bf Ingeniería de Servidores (2015-2016)} \\ Grado en Ingeniería Informática \\ Universidad de Granada} \\ [25pt] % Your university, school and/or department name(s)
\horrule{0.5pt} \\[0.4cm] % Thin top horizontal rule
\huge Memoria Práctica 5 \\ % The assignment title
\horrule{2pt} \\[0.5cm] % Thick bottom horizontal 
}

\author{Antonio Javier Cabrera Gutiérrez } % Nombre y apellidos

\date{\normalsize\today} % Incluye la fecha actual

%----------------------------------------------------------------------------------------
% DOCUMENTO
%----------------------------------------------------------------------------------------

\begin{document}

\maketitle % Muestra el Título

\newpage %inserta un salto de página

\tableofcontents % para generar el índice de contenidos

\listoffigures
\listoftables


\newpage

\section{Al modificar los valores del kernel de este modo, no logramos que persistan después de reiniciar la maquina. ¿ Que archivo hay que editar para que estos cambios sean persistentes?}
Para que los cambios sean permanentes los cambios que hagamos con sysctl debemos de hacerlos tambien en el archivo /etc/sysctl.conf. Por ejemplo, si queremos cambiar la variable panic del kernel podemos hacerlo con \textit{sysctl -w kernel.panic=5} ahora tendriamos que añadir esa linea al archivo /etc/sysctl.conf con \begin{verbatim}echo "kernel.panic=5" >> /etc/sysctl.conf 
\end{verbatim} Con esto estamos indicando que ante un panic el sistema se reinicie a los 5 segundos. \footnote{\url{http://rm-rf.es/sysctl-y-procsys-modificar-parametros-de-kernel/}}
Aquí se puede ver los pasos hechos y la comprobación de que se han realizado correctamente:
\begin{figure}[H] 
\centering
\includegraphics[scale=0.5]{kernel.png}  
\label{figura1:}
\caption{Cambio del parametro panic del kernel}
\end{figure}
\section{¿Con que opción se muestran todos los parámetros modificables en tiempo de ejecución? Elija dos parámetros y explique, en dos lineas, que función tienen.}
Para poder ver los parámetros que se pueden modificar se hace con sysctl -a, yo he elegido dos, pero voy a elegir un tercero ya que el segundo es parecido al primero.
\begin{itemize}
\item dev.raid.speed\_limit\_max: Este parámetro establece la velocidad máxima de reconstrucción de un RAID, el valor por defecto es 200000 y esta expresado en kibibytes por segundo.
\item dev.raid.speed\_limit\_min: Este parámetro establece la velocidad mínima de reconstrucción de un RAID, el valor por defecto es 1000 y esta expresado en kibibytes por segundo.\footnote{\url{http://www.cyberciti.biz/tips/linux-raid-increase-resync-rebuild-speed.html}}
\item kernel.pid\_max: Este parámetro indica el numero máximo de valor que puede tener el PID de un proceso, el valor por defecto es 32768.\footnote{\url{http://www.cyberciti.biz/tips/howto-linux-increase-pid-limits.html}}
\end{itemize}
\section{Realice una copia de seguridad del registro y restaurela, ilustre el proceso con capturas}
Para hacer una copia de seguridad del registro debemos de abrir el regedit para ello pulsamos la tecla Windows + R y en ejecutar escribimos regedit.
\begin{figure}[H] 
\centering
\includegraphics[scale=0.5]{regedit.png}  
\label{figura2:}
\caption{Abrir Regedit}
\end{figure}
El siguiente paso es darle a archivo exportar y elegimos donde exportar el archivo de la copia de registro.
\begin{figure}[H] 
\centering
\includegraphics[scale=0.5]{regedit1.png}  
\label{figura3:}
\caption{Copia del registro}
\end{figure}
Para importar el registro debemos de hacer lo siguiente:
Reiniciamos la maquina y justo cuando arranca rápidamente le damos a la tecla F8 y nos saldrá esta pantalla:
\begin{figure}[H] 
\centering
\includegraphics[scale=0.4]{regedit2.png}  
\label{figura4:}
\caption{Restaurando el registro}
\end{figure}
Le damos a la opción Last Know Good Configuration (advanced)\footnote{\url{https://technet.microsoft.com/en-us/library/cc772043.aspx}}
\section{¿Como se abre una consola en Windows? ¿Que comando hay que ejecutar para editar el registro? Muestre su ejecución con capturas de pantalla}
Para abrir una consola la forma mas cómoda es ejecutar la combinación de teclas \textit{Tecla de Windows + R} y en la ventana que nos aparece escribimos 'cmd'.
Para editar el registro a traves de la consola debemos de escribir REG, aqui vemos lo que nos mostraría cuando ejecutamos REG/?
\begin{figure}[H] 
\centering
\includegraphics[scale=0.5]{reg.png}  
\label{figura5:}
\caption{Uso de la orden REG}
\end{figure}
\section{Las cadenas de caracteres y valores numéricos tienen distintos tipos. Busque en la documentación de Microsoft y liste todos los tipos de valores}
En \footnote{\url{https://support.microsoft.com/es-es/kb/256986}}
podemos encontrar los tipos de datos:
\begin{itemize}
\item Valor binario REG\_BINARY: Datos binarios sin formato. La información sobre el hardware se guarda en forma de datos binarios y se muestra en hexadecimal en el editor del registro
\item Valor DWORD REG\_DWORD: datos representados por un numero de 4 bytes de longitud. Parámetros de controladores de dispositivos y servicios.
\item Valor alfanumérico expandible REG\_EXPAND\_SZ: Cadena de datos de longitud variable.
\item Valor de cadena múltiple REG\_MULTI\_SZ: Valores que contienen listas o valores múltiples.
\item Valor de cadena REG\_SZ: Cadena de texto de longitud fija.
\item Valor binario REG\_RESOURCE\_LIST: Serie de matrices anidada diseñada para almacenar una lista de recursos utilizados por el controlador de un dispositivo hardware o uno de los dispositivos físicos que controla. 
\item Valor binario REG\_RESOURCE\_REQUIREMENTS\_LIST Serie de matrices anidada diseñadas para almacenar una lista de controladores de dispositivo de posibles recursos hardware que el controlador puede utilizar.
\item Valor binario REG\_FULL\_RESOURCE\_DESCRIPTOR: Serie de matrices anidadas diseñada para almacenar una lista de recursos utilizados por un dispositivo físico.
\item REG\_NONE: Datos sin ningún tipo en particular. El sistema o una aplicación escribe estos datos en el registro y los muestra en el editor del registro en formato hexadecimal.
\item Vinculo REG\_LINK: Cadena Unicode que da nombre a un vinculo simbólico.
\item Valor QWORD REG\_QWORD: Datos representados por un numero entero de 64 bytes.
\end{itemize}
\section{Enumere que elementos se pueden configurar en Apache y en IIS para que Moodle funcione mejor}
Para Apache las recomendaciones que nos dan en \footnote{\url{https://docs.moodle.org/23/en/Performance_recommendations}}
son las siguientes:
\begin{itemize}
\item Ajustar la directiva MaxClientes en función de la memoria disponible
\item Reducir el numero de módulos necesarios que apache carga en httpd.conf para reducir el uso de la memoria.
\item Usar la version 2 de apache que tiene el modelo de memoria mejorada que reduce el uso de memoria adicional.
\item En los sistemas Unix bajar el MaxRequestsPerChild en httpd.conf hasta un mínimo de 20 a 30.
\item Para un servidor muy cargado instalar KeepAliveOff o bajando la KeepAliveTimeout a entre 2 y 5, como alternativa se puede instalar un servidor proxy inverso delante del servidor Moodle para almacenar en cache los archivos HTML con imágenes. 
\item Si no se utiliza un archivo .htaccess, establecer la variable de AllowOverride a NONE para evitar búsquedas en .htaccess.
\item Establecer el DirectoryIndex correctamente para evitar la negociación de contenido indicando el archivo index correspondiente.
\item Si no se esta haciendo trabajo de desarrollo en el servidor establecer ExtendedStatus Off y desactivar mod\_info así como mod\_status.
\item Dejar HostnameLookups Off para reducir la latencia de DNS.
\item Considerar la posibilidad de reducir el valor de tiempo de espera entre 30 a 60
\item Para la directiva Options, evitar Options Multiviews ya que realiza una exploración del directorio. Con eso evitamos el uso de entradas y salidas a disco.
\end{itemize}
Para IIS:
\begin{itemize}
\item Establecer el parámetro ListenBackLog entre 2 y 5.
\item Cambiar el valor MemCacheSize para ajustar la cantidad de memoria que IIS utilizara para su cache de archivos.
\item Cambiar el MaxCachedFileSize para ajustar el tamaño máximo de un archivo almacenado en cache.
\item Crear un valor DWORD llamado ObjectCacheTTL para cambiar la cantidad de tiempo que los objetos en la cache se mantienen en la memoria.
\end{itemize}
\section{Ajuste la compresión en el servidor y analice su comportamiento usando varios valores para el tamaño de archivo a partir del cual comprimir. Para comprobar que se esta comprimiendo puede usar el navegador o comandos como curl (see url) o lynx. Muestre capturas de pantalla de todo el proceso.}
Para activar la compresión debemos de acceder al IIS y en nuestro servidor le damos al icono de compresión.
\begin{figure}[H] 
\centering
\includegraphics[scale=0.5]{iis.png}  
\label{figura6:}
\caption{IIS}
\end{figure} 
Ponemos la compresión dinámica y estática sin marcar, le damos a aplicar y reiniciamos el servidor.
\begin{figure}[H] 
\centering
\includegraphics[scale=0.5]{iis1.png}  
\label{figura7:}
\caption{IIS sin comprimir}
\end{figure}
Para comprobar si comprime o no podemos hacerlo con la orden curl -I --compressed y la url para que nos muestre la cabecera y si ha comprimido o no, el resultado es este:
\begin{figure}[H] 
\centering
\includegraphics[scale=0.5]{iis2.png}  
\label{figura8:}
\caption{Comprobacion curl sin comprimir}
\end{figure}
También he comprobado con el navegador firefox con firebug:
\begin{figure}[H] 
\centering
\includegraphics[scale=0.5]{iis3.png}  
\label{figura9:}
\caption{Firebug sin compresión}
\end{figure}
Como se puede ver el archivo ocupa 689 bytes y no muestra ninguna compresión. Vamos a activar la compresión para ver los cambios.
Activamos la compresión estática y dinámica en IIS le damos a aplicar y reiniciamos el servidor.
\begin{figure}[H] 
\centering
\includegraphics[scale=0.5]{iis4.png}  
\label{figura10:}
\caption{Activar compresión IIS}
\end{figure}
Comprobamos con la orden curl igual que antes:
\begin{figure}[H] 
\centering
\includegraphics[scale=0.5]{iis5.png}  
\label{figura11:}
\caption{Curl con compresión}
\end{figure}
También lo probamos con el firebug:
\begin{figure}[H] 
\centering
\includegraphics[scale=0.5]{iis6.png}  
\label{figura12:}
\caption{Firebug con compresión}
\end{figure}
Como podemos ver ahora el archivo ocupa 594 bytes, menos que antes y ahora aparece la cabecera Content-Encoding con valor gzip por lo que podemos comprobar que esta comprimiendo.
\section{Usted parte de un SO con ciertos parámetros definidos en la instalación (Práctica 1), ya sabe instalar servicios (Práctica 2) y cómo monitorizarlos (Práctica 3) cuando los somete a cargas (Práctica 4). Al igual que ha visto cómo se puede mejorar un servidor web (Práctica 5 Sección 3.1), elija un servicio (el que usted quiera) y modifique un parámetro para mejorar su comportamiento. (9.b) Monitorice el servicio antes y después de la modificación del parámetro aplicando cargas al sistema (antes y después) mostrando los resultados de la monitorización.}
Yo he decidido optimizar el servicio MySQL, ya que existen diversos motores de base de datos usados por MySQL, los dos motores mas usados en bases de datos de desarrollo web son MyISAM e InnoDB, depende cual escojamos nos proporcionara un mayor rendimiento.
Ahora bien para medir el rendimiento he utilizado mysqlslap que permite estresar y hacer benchmarking del servidor MySQL. MySQL trae por defecto el motor InnoDB.
\begin{figure}[H] 
\centering
\includegraphics[scale=0.4]{sql.png}  
\label{figura13:}
\caption{Motores de MySQL}
\end{figure}
Para ver esto solo hay que entrar en MySQL poner \textit{use information\_schema;} y a continuacion \textit{select * from ENGINES;}
Para realizar el benchmark ponemos la orden mysqlslap como se indica en la siguiente figura:
\begin{figure}[H] 
\centering
\includegraphics[scale=0.4]{sql1.png}  
\label{figura14:}
\caption{Benchmark para InnoDB}
\end{figure}
Esta orden\footnote{\url{http://dev.mysql.com/doc/refman/5.7/en/mysqlslap.html}} con esos parámetros genera 10000 consultas y ejecuta 100 clientes de forma concurrente y utilizando el motor InnoDB y lo repite 10 veces. Como se puede ver los datos de salida el tiempo medio en segundos que ha tardado en realizar todas las consultas es de 12.547. Ahora realizamos lo mismo pero cambiamos el motor por MyIsam.
\begin{figure}[H] 
\centering
\includegraphics[scale=0.4]{sql2.png}  
\label{figura15:}
\caption{Benchmark para MyIsam}
\end{figure}
Como se puede ver el tiempo medio es de 8.166 segundos utilizando los mismos parámetros.
Por lo que parece que el motor MyIsam es mucho mas rápido que InnoDB\footnote{\url{http://blog.arsys.es/myisam-o-innodb-elige-tu-motor-de-almacenamiento-mysql/}}, esto se debe ya que MyIsam tiene mayor velocidad a la hora de recuperar datos y no hace comprobaciones de la integridad referencial, ni bloquea las tablas para realizar las operaciones, esto hace que tenga una mayor velocidad. Además MyIsam consume menos recursos de memoria por lo que en equipos con problemas en este tipo es recomendable usar MyIsam.
\section{Realice lo mismo que en la cuestión 8 pero para otro servicio}
Esta vez voy a intentar optimizar la velocidad del disco a la hora de realizar una operación en el, ya que con esto optimizaríamos las operaciones realizadas en el RAID, para ello vamos a utilizar la herramienta hdparm. \footnote{\url{http://manpages.ubuntu.com/manpages/natty/man8/hdparm.8.html}
}Esta herramienta tiene múltiple opciones para modificar parámetros del disco. Además trae una opción que mide la velocidad de lectura en cache y en disco.
Antes de empezar a realizar cambios medimos las prestaciones con esta orden:
\begin{figure}[H] 
\centering
\includegraphics[scale=0.5]{disc.png}  
\label{figura16:}
\caption{Medidas de prestaciones antes de realizar cambios}
\end{figure}
Con la letra T hace que se haga un test de la cache y con la t hace una prueba de acceso a disco. Vemos que empezamos con 78.89 MB/sec. Vamos a ver si conseguimos mejorarlo.
Ahora vamos a proceder a cambiar el acceso a disco y lo ponemos a 32 bits. esto se hace poniendo el flag -c1, con un 1 lo activamos y con 0 lo desactivamos. Vemos en la siguiente imagen como aumenta a 82.66 MB/sec. A continuación establecemos el modo \textit{read ahead} este modo acelera las lecturas secuenciales.\footnote{\url{http://dmolinap.blogspot.com.es/2008/05/optimizando-el-disco-duro-con-ubuntu.html}} para ello usamos el flag -A1 para activarlo.
\begin{figure}[H] 
\centering
\includegraphics[scale=0.5]{disc1.png}  
\label{figura17:}
\caption{Medidas de prestaciones con dos modificaciones}
\end{figure}
Como vemos tras esta modificación ya hemos conseguido un tiempo de lecturas de disco de 84.75 MB/sec.
Por ultimo vamos a activar el ultra dma (UDMA,Ultra Direct Memory Acces) esto hace que la tasa de transferencia se eleve.
\begin{figure}[H] 
\centering
\includegraphics[scale=0.5]{disc2.png}  
\label{figura18:}
\caption{Medidas de prestaciones con tres modificaciones}
\end{figure}
Como podemos ver tasa de transferencia ha subido a 85.76 MB/sec. Estas son las optimizaciones que he realizado, he intentado hacer algunas mas pero mi disco no admitía algunas modificaciones en ciertos parámetros.
Para que estas modificaciones se establezcan cada vez que arranca el sistema habría que añadir la orden \textit{hdparm -X66 -c1 -A1 /dev/sda} en los ficheros de arranque.\footnote{\url{http://www.espaciolinux.com/2003/07/acelerando-el-acceso-a-disco-en-linux-con-hdparm/}}Con estos cambios al hacer mas rápido las operaciones de lectura estaríamos optimizando nuestro RAID.
\end{document}